% Text for MPD Run 16 Rwt 3

\section{Introduction}

This Appendix describes the results from the mode of the posterior distribution (MPD) (to compare model estimates to observations), diagnostics of the Markov chain Monte Carlo (MCMC) results, and the MCMC results for the estimated parameters.  The final advice and major outputs are obtained from the MCMC results. Estimates of major quantities and advice to management (such as decision tables) are also presented in the main text.

% Taking next section of text from the Sweave.

\section{Mode of the posterior distribution (MPD) results}

Awatea first determines the MPD for each estimated parameter.  These are then used as the starting points for the MCMC simulations. The MPD fits are shown for the survey indices (Figure~\ref{fig:survIndSer2}), the commercial catch-at-age data (as overlaid age structures in Figures \ref{fig:ageCommFemale1} and \ref{fig:ageCommMale1}), the West Coast Haida Gwaii (WCHG) synoptic survey (Figure~\ref{fig:ageSurvWCHGSynopticFemale1}), the Hecate Strait (HS) synoptic survey (Figure~\ref{fig:ageSurvHSSynopticFemale2}), the Queen Charlotte Sound (QCS) synoptic survey (Figure~\ref{fig:ageSurvQCSoundSynopticFemale3}), and the West Coast Vancouver Island (WCVI) synoptic survey series age data (Figure \ref{fig:ageSurvWCVISynopticFemale4}). The results are sensible and are able to capture the main features of the data sets fairly well. There appears to be relative consistency between the available data sources.

Residuals to the MPD model fits are provided for the six survey indices (Figures \ref{fig:survResWCHGSynoptic} to \ref{fig:survResUSTriennial}), and the five sets of age data (Figures \ref{fig:commAgeResids} and \ref{fig:survAgeResSer4}). These further suggest that the model fits are consistent with the data, as do the mean ages for the two sets of age data (Figure \ref{fig:meanAge}).

Figure \ref{fig:stockRecruit} shows the resulting stock-recruitment function and the MPD values of recruitment over time (though see Figure \ref{fig:recruitsMCMC} for the MCMC values of recruitment). Figure \ref{fig:recDev} shows that the recruitment deviations display no trend over time, and that the auto-correlation function of the deviations appears satisfactory. Figure \ref{fig:selectivity} gives the MPD fits for the selectivities, together with ogive for female maturity. Figure~\ref{fig:exploit} gives the exploitation over time. The values of the log-likelihood and objective functions for the MPD fits are given in Table \ref{tab:like}.


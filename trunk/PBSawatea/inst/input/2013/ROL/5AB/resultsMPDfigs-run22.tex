% MPD Figures for Run 22 Rwt3

\onefig{survIndSer2}{Survey index values (points) with 95\% confidence intervals (bars) and MPD model fits (curves) for the fishery-independent survey series.}

\onefig{CPUEser}{ CPUE index series, 95\% error bars are based on lognormal assumption (double check error bars).}

\onefig{ageCommUnisex1}{Observed and predicted commercial proportions-at-age for females. Note that years are not consecutive.}

\clearpage 

\onefig{ageSurvQCSSynopticUnisex1}{Observed and predicted proportions-at-age for QCSSynoptic survey.}

\onefig{survResQCSSynoptic}{Residuals of fits of model to QCSSynoptic survey series (MPD values). Vertical axes are standardised residuals. The three plots show, respectively, residuals by year of index, residuals relative to predicted index, and normal quantile-quantile plot for residuals (horizontal lines give 5, 25, 50, 75 and 95 percentiles).}

\onefig{commAgeResids}{Residual of fits of model to commercial proportions-at-age data (MPD values).  Vertical axes are standardised residuals. Boxplots show, respectively, residuals by age class, by year of data, and by year of birth (following a cohort through time). Boxes give interquartile ranges, with bold lines representing medians and whiskers extending to the most extreme data point that is $<$1.5 times the interquartile range from the box. Bottom panel is the normal quantile-quantile plot for residuals, with the 1:1 line, though residuals are not expected to be normally distributed because of the likelihood function used; horizontal lines give the 5, 25, 50, 75, and 95 percentiles (for the total of 220 residuals).}

\clearpage 

\onefig{survAgeResSer1}{Residuals of fits of model to proportions-at-age data (MPD values) from QCSSynoptic survey series. Details as for Figure \ref{fig:commAgeResids}, for a total of *** residuals.} % number of years of survey age data * (number age classes -1) * 2

\twofig{recDev}{recDevAcf}{Top: log of the annual recruitment deviations, $\epsilon_t$, where bias-corrected multiplicative deviation is  $\mbox{e}^{\epsilon_t - \sigma_R^2/2}$ where $\epsilon_t \sim \mbox{Normal}(0, \sigma_R^2)$. Bottom: Auto-correlation function of the logged recruitment deviations ($\epsilon_t$), for years 1984-2009 (determined as the first year of commercial age data minus the accumulator age class plus the age for which commercial selectivity for females is 0.5, to the final year that recruitments are calculated minus the age for which commercial selectivity for females is 0.5).}



% Rock Sole 5CD base case (Run08)
\section{Bayesian MCMC Results}

The MCMC procedure performed 50,000,000 iterations, sampling every 50,000$^\mathrm{th}$ to give 1,000 MCMC samples.  The 1,000 samples were used with no burn-in period (because the MCMC searches started from the MPD values). The quantiles (0.05, 0.50, 0.95) for estimated parameters and derived quantities appear in Tables~\ref{tab:MCMCpar} and \ref{tab:MCMCderived}. In particular, the current year median estimate of $B_{\Sexpr{currYear}}$ is \Sexpr{format(median(Bcurr.MCMC),digits=0,big.mark=",",scientific=FALSE)}~t. The median depletion estimate $B_{\Sexpr{currYear}}/B_0$ is  \Sexpr{round(median(Bcurr.MCMC / B0.MCMC),3)}.

MCMC traces show acceptable convergence properties (no trend with increasing sample number) for the estimated parameters (Figure \ref{fig:traceParams}), as does a diagnostic analysis that splits the samples into three segments (Figure \ref{fig:splitChain}). Most of the parameters (e.g., $R_0$) move from the initial MPD estimate to some other median value.  Pairs plots of the estimated parameters (starting at Figure~\ref{fig:pairs1}) show no undesirable correlations between parameters.  In particular, steepness, $h$, and the natural mortality parameter, $M_1$\Sexpr{if (Nsex>1) paste(" (and $M_2$ if a two-sex model)")}, show little correlation, suggesting there are sufficient data to estimate them simultaneously. Thus, the MCMC computations seem satisfactory.

Marginal posterior distributions and corresponding priors for the estimated parameters are shown in Figure \ref{fig:pdfParameters}. For most parameters, with the exceptionf $h$, it appears that there is enough information in the data to move the posterior distribution away from the prior. The estimate of natural mortality, $M_1$, shifted significantly higher from 0.20 to 0.25 while the $h$ posterior basically mirrored the prior. Corresponding summary statistics for the estimated parameters are given in Table \ref{tab:MCMCpar}.

<<Bcalcs, results=hide, echo=FALSE>>=
BoverB0 = currentMCMC$B/currentMCMC$B[,1]
BoverB0.med = apply(BoverB0,2,median)
BoverB0.med.min = min(BoverB0.med)
BoverB0.med.min.yr = names(BoverB0.med)[grep(BoverB0.med.min,BoverB0.med)][1]
@

The marginal posterior distribution of vulnerable biomass and catch (Figure~\ref{fig:VBcatch}) shows a decline in the population from 1955 to aprroximately 1980, a levelling off during the 1980s and 1990s, followed by an increase from 2000 until the final year (\Sexpr{rev(years)[1]}). 
The median spawning biomass relative to unfished equilibrium values (Figure~\ref{fig:BVBnorm}) reached a minimum of \Sexpr{round(BoverB0.med.min,3)} in \Sexpr{BoverB0.med.min.yr} and currently sits at \Sexpr{round(rev(BoverB0.med)[1],3)}. 
The recruitment patterns for \Sexpr{area.name} Rock Sole show occasional upticks in 1989, 2000, and 2008 (Figure~\ref{fig:recruitsMCMC}). 
Exploitation rates were elevated during three periods 1966-71, 1975-80, and 1988-95, where the latter period saw rates on the order of 30\% (Figure~\ref{fig:exploitMCMC}).
A phase plot showing the time-evolution of spawning biomass and exploitation rate relative to $B_\mathrm{MSY}$ and $F_\mathrm{MSY}$ (Figure~\ref{fig:snail}) show a meandering within a good zone (low exploitation, high biomass).

\section{Projection results and decision tables}

Projections were made to evaluate the future behaviour of the population under different levels of constant catch, given the model assumptions.  The projections, starting with the biomass at the beginning of \Sexpr{currYear}, were made over a range of constant catch strategies (\Sexpr{min(as.numeric(row.names(refProbs$LRP)))}-\Sexpr{prettyNum(max(as.numeric(row.names(refProbs$LRP))), big.mark=",")}~t) for each of the \numMCMC~MCMC samples in the posterior, generating future biomass trends by assuming random recruitment deviations.  Future recruitments were generated through the stock-recruitment function using recruitment deviations drawn randomly from a lognormal distribution with zero mean and constant standard deviation (see Appendix F for full details). Projections were made for \Sexpr{dim(refProbs$LRP)[2]-1} years. This time frame was considered to be long enough to satisify the 'long-term' requirement of the Request for Science Information and Advice (Appendix A), yet short enough for the projected recruitments to be mainly based on individuals spawned before \Sexpr{currYear} (and hence already estimated by the model).

Resulting projections of spawning biomass are shown for selected catch strategies (Figure \ref{fig:Bproj}). These suggest that the recent increase in spawning biomass would most likely continue for a catch of \Sexpr{onePolicy}~t, which is \Sexpr{paste(ifelse(onePolicy>recentCatchMean," larger "," smaller "))} than the recent average catch of \Sexpr{round(recentCatchMean)}~t. 

Note that recruitment is drawn from the estimated stock-recruitment curve with lognormal error that has a standard deviation of \Sexpr{sigmaR} and a mean of zero. However, this approach of average recruitment does not accurately simulate the occasional large recruitment events that have occurred for this stock (Figure~\ref{fig:recruitsMCMC}).

Decision tables give the probabilities of the spawning biomass exceeding the reference points in specified years, calculated by counting the proportion of MCMC samples for which the biomass exceeded the given reference point.

Results for the three $\Bmsy$-based reference points are presented in Tables \ref{tab:LRP}-\ref{tab:Bmsy}. For example, the estimated probability that the stock is in the provisional healthy zone in 2017 under a constant catch strategy of 1,000~t is P$(B_{2017} > 0.8 \Bmsy)=\Sexpr{round(refProbs$URP["1000","2017"], dig=2)}$ (row '1000' and column '2017' in Table \ref{tab:URP}). 

Table~\ref{tab:Bcurr} provides probabilities that projected spawning biomass $B_t$ will exceed the current-year biomass $B_{\Sexpr{currYear}}$ at the various catch levels. The first column populated by zero values simply means that the current-year biomass will never be greater than itself. Table~\ref{tab:umsy} shows the probabilities of projected exploitation rate $u_t$ exceeding that at MSY ($u_\mathrm{MSY}$).

For the maximum sustainable yield (MSY) calculations, projections were run for \Sexpr{currentMSY$maxUind[1]} values of constant exploitation rate $u_t$ between \Sexpr{currentMSY$uMin[1]} and \Sexpr{currentMSY$uMax[1]}, until an equilibrium yield was reached within a tolerance of \Sexpr{currentMSY$tolerance[1]}~t (or until \Sexpr{prettyNum(currentMSY$maxProj[1], big.mark=",")} years had been reached). This was done for each of the \numMCMC~samples.
The lower bound of $u_t$ was reached for \Sexpr{ifelse(sum(currentMSY$imsy==1)==0," none ",sum(currentMSY$imsy==1))} of the MCMC samples, and the upper bound was reached by \Sexpr{ifelse(sum(currentMSY$imsy==currentMSY$maxUind[1])==0," none ",sum(currentMSY$imsy==currentMSY$maxUind[1]))} of the samples.
Of the \Sexpr{prettyNum(currentMSY$maxUind[1] * num.MCMC, big.mark=",")} projection calculations, \Sexpr{ifelse(sum(currentMSY$nProjMatTF)==0," all converged ",paste(sum(currentMSY$nProjMatTF)," did not converge ",sep=""))} by \Sexpr{prettyNum(currentMSY$maxProj[1], big.mark=",")} years.

The most recent Rock Sole assessment (Starr et al. 2006. Rock sole (\emph{Lepidopsetta} spp) in British Columbia, Canada: Stock Assessment for 2005 and Advice to Managers for 2006/2007. PSARC Working Paper, Unpublished Manuscript, Available from K. Holt, Fisheries and Oceans Canada) used historical reference points -- (i)~a limit reference point based on the minimum biomass during a period when the biomass is experiencing depressed levels, and (ii)~a target reference point based on the mean biomass during a period when biomass was considered stable and sustainable after fishing. For \Sexpr{area.name}, the 2006 assessment recommended that the limit biomass be determined from the years \Sexpr{paste(range(HRP.YRS[["blimYrs"]]),collapse="-")} and that the target biomass be calculated as the mean over the years \Sexpr{paste(range(HRP.YRS[["btarYrs"]]),collapse="-")}. We have retained these histroical reference points for this assessment. Table~\ref{tab:blimHRP} and Table~\ref{tab:btarHRP} provide probabilities that projected biomass will exceed these historical limit and target reference points, respectively. Similarly, Table~\ref{tab:utarHRP} provides probabilities that the projected exploitation rate $u_t$ exceeds a mean exploitation rate over the years \Sexpr{paste(range(HRP.YRS[["utarYrs"]]),collapse="-")}.

% End of Appendix G


